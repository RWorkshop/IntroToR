
As it outside the scope of this research, which shall focus on the domain of method comparison studies, an exploration of Gauge R\&R will be limited to a short summary.

In this thesis, we will not actively engage with subject of Gauge R\&R. However, where there are very conspicuous overlaps in subject mattter (for example, repeatability and multiple factors), gauge R\&R 
will be reverted to, so as to provide a proper statement of findings, which may open avenues of further research of the application of statistical methods in different domains.

A key aspect of Gage $R\&R$ is that consideration of multiple factors that may affect measurement quality (e.g. users, time of day, machine etc). An interesting parallel is found in the work of Laio and Shaio, who depart from the conventional formulation of the method comparison problem by attempting to include factors other than the method of measurement.

For many of the techniques commonly used for the method comparison problem, extra variables are not considered (with the exception of formal models such as Dunn). This is for various reasons; the formulation is not feasible, or a solution is not tractable.  A key exception to this is the Structural Equation Approach as proposed by Dunn.

The topics of Structural Equation Modelling is well described in literature, and for the sake of brevity, we will again limit ourselves to a brief summary.

Crucially several of the key requirements of the SEM approach as not feasible in many day-to-day method comparison studie problems. As discussed earlier, method comparison studies are prevalent in clinical research, where in many cases, the availability of measurement data is limited.

Another aspect that mitigates against the use of SEM, is the requirement of an advanced statistical skillset. Although many "non-statisticians" ( to borrow Bland and Altman's phrase) have developed an appreciable 
statistical skills base, many advanced methods may be beyond what should be reasonably expected of them. This is a matter commented upon in Bland and Altman 1983, which further emphasised the ease of use of a method.

In various technological domains, professionals may encounter the notion of a "technology acceptance model". The technology acceptance model theorises why some technologies become prevalent, whereas other technologies do not, despite the fact that they are perfectly adequaate for the task, or perhaps superior.

Perception is a key matter in the technology acceptance model.

