\documentclass{beamer}

\usepackage{graphics}
\usepackage{amsmath}
\usepackage{framed}


\begin{document}

%------------------------------------%
\begin{frame}
\frametitle{\texttt{R} Computing}
{
\Huge
\[ \mbox{ \texttt{R} Computing }\]
}

{
\Large
\[ \mbox{ Integers and Numerics }\]
}



\end{frame}
%------------------------------------%
\begin{frame}
\frametitle{\texttt{R} Computing}
\texttt{R} has no single precision data type. All \textit{\textbf{real}} numbers are stored in double precision format.
\texttt{is.numeric} is a more general test of an object being interpretable as numbers.
\begin{itemize}
\item \texttt{is.numeric()} : double-precision vector.
\item \texttt{is.integer()} : double-precision vector.
\item \texttt{is.double()} : double-precision vector.
\end{itemize}

\end{frame}

%------------------------------------%
\begin{frame}
\frametitle{\texttt{R} Computing}
\begin{itemize}
\item
\item
\end{itemize}
\end{frame}
%------------------------------------%
\end{document}
