\section{Flood Risk Management}
\subsection{Indicative Floodplain Map}

A map that delineates the areas estimated to be at risk of flooding during an event of specified flood
probability. Being indicative, such maps only give an indication of the areas at risk but, due to the scale and
complexity of the exercise, cannot be relied upon to give precise information in relation to individual sites.

\subsection{ Likelihood of flooding}
A general concept relating to the chance of an event occurring. Likelihood is generally expressed as a probability or a frequency of a flood of a given
magnitude or severity occurring or being exceeded in any given year. It is based on the average frequency estimated, measured or extrapolated from records over a large number of years and is usually expressed as the chance of a particular flood level being exceeded
in any one year. For example, a 1 in 100 or $1\%$ flood is that which would, on average, be expected to occur
once in 100 years, though it could happen at any time.

\subsection{ Flood zones}

The flood zones are defined on the basis of the probability of flooding from
rivers and the sea. Because of the generally more dynamic nature of coastal
flooding compared to river flooding, a lower probability of coastal flooding is
used to define the highest-risk zone.
\begin{itemize}
\item Zone A is at highest risk and has a 1 in 100 (or $1\%$) chance of flooding in any
one year from rivers and a 1 in 200 (or $0.5\%$) chance of flooding from the
sea.
\item Zone B is at moderate risk of flooding from rivers and the sea and its outer limit
is defined by a 1 in 1000 (or $0.1\%$) chance of flooding in any one year.
\item Zone C is the low risk area, with a less than 1 in 1000 ($<0.1\%$) chance of
flooding from rivers, estuaries or the sea in any one year.
\end{itemize}
The definition of these zones does not, however, take account of the potential
for flooding from other sources, such as ground water or artificial drainage
systems. Flooding from these sources could occur in any of the zones and as
such should always be considered, regardless of zone.

\section{ How is flood risk measured?}
Flood risk is a combination of the likelihood of occurrence and the consequences of a flood occurring. This is normally expressed as:
\[ \mbox{ Flood risk } = \mbox{ probability } \times \mbox { consequences }.\]
Probability is difficult to estimate because it has to take account of the uncertainty of hydrological predictions based on the analysis of many years of flow records. Consequences are also complex to measure in terms of the
potential loss of life, damage to property etc, which depend on the vulnerability of the land-use and property affected by the flood.



%----------------------------------------------------------------------------------------%
\section{EVT for hydrology}
Probability methods have often been used for solving many kinds of hydrological problems. The principal objective of these methods is to find a probabilistic distribtuions, or models, that are capable of explaining the observed  characteristscs of physical phenomenon of interest.


%----------------------------------------------------------------------------------------%
\section{Hydrometry}
Hydrometry: measuring hydrological phenomena
\begin{itemize}
	\item How to measure (areal) rainfall?
	\item How to measure runoff?
\end{itemize}


\section{One-hundred-year floods }
A one-hundred-year flood is calculated to be the level of flood water expected to be equaled or exceeded every 100 years on average. The 100-year flood is more accurately referred to as the $1\%$ annual exceedance probability flood, since it is a flood that has a 1\% chance of being equaled or exceeded in any single
year.Similarly, a flood level expected to be equaled or exceeded every 10 years on average is known as a \textbf{\emph{ten-year
		flood}}.

Based on the expected flood water level, a predicted area of inundation can be mapped out. This floodplain map figures very
importantly in building permits, environmental regulations, and flood insurance.

\section{Time of concentration}
Time of concentration is a concept used in hydrology to measure the response of a watershed to a rain event. It is defined as the
time needed for water to flow from the most remote point in a watershed to the watershed outlet.

It is a function of the topography, geology, and land use within the watershed. Time of concentration is useful in predicting flow
rates that would result from hypothetical storms, which are based on statistically-derived return periods.

For many (often economic) reasons, it is important for engineers and hydrologists to be able to accurately predict the response of a watershed to a given rain event. This can be important for
infrastructure development (design of bridges, culverts, etc.) and management, as well as to assess flood risk.

\section{Cumulative frequency analysis}

Cumulative frequency analysis is the analysis of the frequency of occurrence of values of a phenomenon less than a reference value. The phenomenon may be time or space dependent. Cumulative frequency is also called frequency of non-exceedance.

Cumulative frequency analysis is done to obtain insight into how often a certain phenomenon (feature) is below a certain value.

This may help in describing or explaining a situation in which the phenomenon is involved, or in planning interventions, for example
in flood protection.
%----------------------------------------------------------------------------------------%

